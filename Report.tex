\documentclass[10pt]{article}
\usepackage[utf8]{inputenc}
\usepackage{graphicx}
\usepackage[hidelinks]{hyperref}
\usepackage[a4paper,top=1in,bottom=1in,right=1in,left=1in]{geometry}
\usepackage{fancyhdr}%to use header and footer in file
\usepackage{caption}
\usepackage{subcaption}
\usepackage{float}
\graphicspath{{images/}}
\graphicspath{ {figures/} }

\rfoot{Page \thepage}%to diaplay page number as footer
\lhead{}%to display shell scripting as left header

\pagestyle{fancy}
\fancyhf{}

\title{}
\author{}

\date{\today}
\pagestyle{fancy} % Turn on the style
\fancyhf{} % Start with clearing everything in the header and footer
% Set the right side of the footer to be the page number
\fancyfoot[R]{\thepage}

\begin{document}

\pagenumbering{gobble}% to remove numbering from title page

\begin{titlepage}

\pagenumbering{arabic} %to add numbering
\centering
\vfill

\textbf{\Huge{Telecommunication Software Lab   %textbf makes the text bold and Huge make the font size large
\linebreak   %to introduce line break
\\ELP 718}}\\
\rule{\textwidth}{3pt}  %to introduce a red coloured bar
\vskip2cm
\textbf{\Large{\emph{Assignment No.7 Report}}}\\
\vskip 0.5cm
\textbf{\Large{\emph{Python}}}
\vskip1cm
\emph{Submitted By:-}
\vskip0.1cm
\Large{\textbf{\\\emph{Nitin Garg}}}
\vskip0.1cm
\Large{\textbf{\\\emph{Entry No - 2016JTM2079}}}
\vskip3cm
\begin{figure}[hbtp]

\begin{center}
\includegraphics[scale=0.05]{iitd_logo.png}
\end{center}
\end{figure}

\vskip1cm
\textbf{\emph{Indian Institute of Technology}}\\
\textbf{\emph{Delhi}}\\
\vskip0.5cm     %introduce a space of 0.5cm between two lines 
September 12,2016
\vfill  %intoduces vertical space 
\end{titlepage}






\pagebreak



\newpage



\tableofcontents




\newpage

\section{Introduction:}
Python is a high level, interpreted, interactive and object oriented scripting
language. Python is designed to be highly readable It
uses English keywords
frequently where as other languages use punctuation
and it has fewer syntactical
constructions than other languages.
\begin{itemize}
\item{Python is Interpreted:
Python is processed at runtime by the interpreter You do not need to compile your program before executing it. This is similar to PERL and PHP.}
\item{Python is Interactive:You can actually sit at a Python prompt and interact with the interpreter directly to write your programs}
\item{Python is Object -Oriented:Python supports Object
Oriented style or technique of programming that encapsulates code within objects.}
\item{Python is a Beginner's Language: Python is a great language for the beginner-
level programmers and supports the development of a wide range of applications from simple text processing to WWW browsers to games}
\end{itemize}



\textbf{Python Features:}
\bigskip

 

\begin{itemize}
\item{ Easy to learn}
\item{A broad standard library}
\item{Interactive Mode}
\item{ GUI Programmming available}

\end{itemize} 








\newpage

\section{Problem Statement}
\subsection{Problem Statement 1}
Write a Python program that can take a big string (with spaces) as input from the command line and count number of times a word occurs in the string and also print the top 3 words in terms of their frequency of count.
Also print the next permutation of each word appearing in the string


\subsubsection{Input Format}
any string iit delhi delhi iit delhi nit nit tjdfjgvnjsd 
   
 
 
 

\subsubsection{Output Format}

delhi iit nit

diehl
\bigskip








\bigskip 

 




\newpage
\section{Problem Statement 2}


You are designing a Graphical user Interface (GUI) to depict the location of a mobile user in a square whose corner points are (1,1) (-1,1) (1,-1)(-1,-1). In real life, the user’s location would come from a database available with the MSC. For the moment, generate the user location using the random function generator function in Python to generate a number between [0,1). 


 
\newpage




\pagebreak
\section{Problem Statement 3}
You have to design an addressing code for a shipping company that works all around India. The address given by the customer is split into fields of
\begin{itemize}
\item{Name}
\item{City}
\item{District}
\item{State/Union Territory}
\end{itemize}
Let's suppose at the intake the employer enters all the above data into the computer, now the coding machine has to build two codes out of the data.

Let's suppose at the intake the employer enters all the above data into the computer, now the coding machine has to build two codes out of the data.

First is machine readable like barcodes, in the form 1’s and 0’s as:



IIT Roorkee  = 001



Roorkee= 010



Uttarakhand = 100



Hence the generated gives the collection center no. CCNO = 100010001





Second is human readable, build by combination of first three letters of a place.



For example :



Prof. Ram Mishra



D - 15, North Enclave



IIT Roorkee, Roorkee


Uttarakhand

Hence human readable code HCCNO =  UTTROOIIT100010001








Create a database with some default addresses.
The database should be editable(Add, delete, modify).
Also notify any discrepancy in data to the employee if the address is invalid or do not exist in the database.


\newpage




\pagebreak
\section{Logic}

\subsection{Problem Statement 1}

In this problem we have to enter a string to which we will convert it into list because processing list in python is much easier than strings.

using standard function for list we will find the most occuring letter in the string.


\bigskip

\subsection{Problem Statement 2}

In this problem we have to take user defined inputs for no of random locations generated.
It is done using importing a library random and math
As we have to calculate no of points lying in the unit circle we will calculate it using finding the magnitude of distance from centre.
And thus calculating the efficiency



\bigskip

\subsection{Problem Statement 3}

In this we have to create a database which we can modify delete or add.

This is done by intializing this database in the form of dictionaries.






\bigskip








\newpage


\section{Screenshots}

\subsection{Problem statement 1}

 \includegraphics[scale=0.6]{2.png} 





\newpage




\subsection{Problem statement 2}

\includegraphics[scale=0.6]{1.png} 





  

\pagebreak
\newpage

\subsection{Problem statement 3}

\includegraphics[scale=0.6]{3.png} 


\pagebreak
\newpage
\section{References}

\begin{thebibliography}{9} %to give refernces

\bibitem{IEEEhowto:kopka}
H.~Kopka and P.~W. Daly, \emph{A Guide to \LaTeX}, 3rd~ed.\hskip 1em plus
  0.5em minus 0.4em\relax Harlow, England: Addison-Wesley, 1999
  
  \bibitem{IEEEhowto:kopka}
MArtin C Brown, \emph{Python : The Complete Reference}, 13th~ed.\hskip 1em plus
  0.5em minus 0.4em\relax India: Apress, 2010
  
 
  \bibitem{IEEEhowto:kopka}  
  
Allen Downey , \emph{Learning with Python}, 4.1 Edition .\hskip 1em plus 0.5em minus 0.4em\relax Dreamtech,2015


 
\end{thebibliography}




\end{document}